\chapter{Introduction} 
Tabletop RPG (role playing game) is a form of game in which the players play, and describe their characters through speech. The players are able to describe, or make, their own characters, their actions, powers, and personality based on a formal system of rules and guidelines \ref{JohnFKim}. Tabletop RPG is hereby an imaginative role-playing game, which puts players in the role of different magical creatures\ref{Banks}.
That been said unlike other role playing games, tabletop RPG sessions are often directed like a verbal movie: only the articulated component of a role is acted out by the players. Here one of the players are designated the role of ‘Game Master’ (GM), who in this case is the director of the movie \ref{JohnFKim}. While everyone sits around a table, the Game Master creates a world, and breathes life into it such that the players are enraptured by the game and their imagination is triggered\ref{Banks}. It is hereby up to the Game Master to weave the strands of the story and plot the course of the adventure in which the players will play (for further clarification see analysis).
During normal play, the only tools the Game Master has to work with are voice acting, roleplaying, and pen/paper. These tools help the players imagine the world that the Game Master is trying to convey and hereby improve the overall game experience \ref{Banks}.  

An important observation is that the user experience of the Tabletop RPG is solely based on the Game Master and his ability to create immersion through the few tools he has available. If the scene is not set properly and the interaction between the players and the Game Master is not optimal the game will loose its ‘fun factor’. 
Taking the above into consideration, many researchers have throughout the years studied emotions and hereby also how to improve user experience. One of the findings in this field includes human senses. 
Researchers have studied human senses and their effect. It has been made clear that humans have five primary senses: Touch, smell, hearing, smell and taste \ref{M. Zanker}.  The five senses collaborate closely to enable the mind into a better understanding of the surroundings \ref{Groeger}. 
That said, some researchers state that presenting the brain with sensory information from different sources not only plays a role in our understanding of the surroundings but that it also plays an integral role in human emotions. Thomson has recently instigated the concept of linking the human senses with emotions. He states that the humans experience with different things such as a game is highly linked to how our sensors are stimulated and how we perceive them \ref{J.Thomson}. 

It has therefore been decided that this paper will investigate how to create a digital multi modal system, which triggers the human senses to increase user experience. In other words: this the paper will focus on enhancing the game experience via new technology-based tools for the GM. 

In accordance with the description above, an initial problem statement would be:

“How can we improve the game experience for all players during a session of tabletop RPG through the use of a digital multimodal system?”  
