\chapter*{Tabletop Role-playing Games}
Tabletop role-playing is a way of playing out stories where the players take on roles as actors while a person acting as what is called the game master sets the stage and the extras of the story. These roles played by the players are referred to a characters. The story is usually progressed with the players' verbal interaction with the game master, inquiring about the world where after they decide on an action of their character. The game master then react to the player interaction with the game world creating new circumstances for the players to react to. 

Sometimes these circumstances are conflicts or challenges to the players' characters that they then have to try and resolve in creative ways within the framework of their characters capabilities. When it is time to resolve the results of the characters efforts there is often a system in place involving chance that decides if the player characters succeed or not. This element of chance is generally simulated through dice, where the players will be given ways to somewhat affect the results of the dice rolls to reflect the characters' abilities. Characters capabilities are often written down on paper or digitally on what is called a character sheet.

\subsection*{Pathfinder}
Pathfinder is a specific rule set for playing tabletop role-playing developed by Paizo Publishing in 2008. The rule set is a conversion from the classic fantasy role-playing system of Dungeons and Dragons. The system of Pathfinder is based on the 3.5 edition of Dungeons and Dragons that was published in 2003 \ref{WotCtimeline}, but was converted with the intention of creating a more streamlined and to some degree more modernised fantasy role-playing experience. \ref{PaizoHome} \ref{PFcore}

\subsection*{Ways to play}
Tabletop role-playing games, such as Pathfinder, can be played in many ways.
One of the ways used as reference in Pathfinder is the use of physical representations on 1 inch square grid game maps to represent distance and the entities in the game, but the game can also be played without this. 
Combat are in both systems a key component and the act of fighting is set-up very much like a board game with pieces moving on the square tiles representing 5 feet in the game world. 
The games can be played very simply or with more complex rules that represent some of the more advanced circumstances that the player characters could encounter. 
There are for example rules in place for how lighting affect visibility and different ways to calculate the characters' health and how that is affected during the characters' adventures. 
The style of play changes from game master to game master and sometimes the same game master can use different styles based on who the players are. 
The maps used during play can both be pre-printed, drawn at the session or constructed with game tiles.
Sound effects and background music can be used to set the mood of the game session or to simulate auditory cues to the players.
Some game masters change the lighting in the game area to represent the lighting of the current area of the players.
There are also scented candles on the market made to represent the smells of different places and scenarios the players may encounter to further push the player's immersion into the game. \ref{KickCandle}

\section*{Player - Game master relation}
Player actions and game master actions in the game are different on one fundamental level. While the players are limited by the game masters rulings, the game master are able to make anything he wishes happen in the game without any influence or limitations from the players. The rules of the game are not there to limit the players or the game master, but to create consistency and create a system wherein both parties can play on an even playing field without the game master having to come up with ways to resolve every possible outcome of the player or their own actions in the game.

Players do not always have all the information of the game world. This is to simulate that just like their characters they do not know everything. The game master knows practically everything in the game except for one thing; how the players are going to make their characters react to different situations and what the result of their efforts will be. The game rules are heavily build on the concept of chance in that every time a player character or non player character do something that have a risk of failure, the result is decided with dice. The game master may fudge with the result to get a desired outcome to progress or direct the narrative in a specific way.

The role of the player and the game master are very different. Their roles are not just different in regards to responsibilities and powers within the game, but also in their goal at the game session. The game master is the facilitator of the game and the players are the participants. 

\subsection*{Player role}
Players can make their characters do anything within the game, but whether or not they succeed at this intend is decided firstly by the rules and finally by the game master. Even though a rule may allow for an action the game master may deem it impossible because of the circumstances that the character is in.

Each player character are different, in that they have different abilities and different personalities. Characters abilities are decided at the start of the game but they also develop over the course of the game. Characters are persistent in that you can use the same character at multiple game sessions and continue developing their story and abilities.

Players have four main categories of actions that they are able to do in the game. Within three of these categories there are more specific actions that they can take that are dictated by the rules of the game. 
\begin{itemize}
\item \textbf{Skill use}

When the players use a skill recorded on their character sheet to do something within the game related to their characters learned abilities.

\item \textbf{Social interaction}

When the player interact with another player or the game master via in-game dialogue. Some game groups acts this out while others states their intend and then use skills from the character sheet to determine the result.

\item \textbf{Combat}

Combat is the most complicated part of Pathfinder. The players are given a list of specific actions that their character can take and the players then decide based on an action economy within the game what they wish to do. Combat are segmented into turns and rounds where a turn represents about 6 seconds within the game world. The players limited actions within their individual turn is to represent the restriction of time.

\item \textbf{Free interaction}

This fourth category is special in that it is action that is not directly dictated by the rules but may still be within the characters capabilities. The rules can not account for everything that a player might imagine their character to do and it is in these instances that it is up to the game master to make their ruling of how this could be done within the scoop of the game world.
\end{itemize}

\subsection*{Game Master role}
The game master's role within the game is to create the game world - populate it with characters, places, creatures and everything that could be imagined in a fantasy world. The game master does not always have to do all of this by himself, he can get help from the players to make up the world or refer to existing world settings or other fiction for inspiration. The game master takes the players on an adventure; an imaginative adventure narrated by the game master with the players acting out the protagonists in collaborative storytelling. To create a story where the players are surprised and have to improvise and make decisions based on the situation within the game, it is a strong tool for the game master to withhold information from the players that their characters do not know yet.
The rulebook of Pathfinder describe the game master as having five roles:
\begin{quote}
"\textbf{Storyteller:} First and foremost, the Game Master is a storyteller. He presents the world and its characters to the players of the game, and it is through the GM that the players interact with them. The Game Master must be able to craft stories and to translate them into a verbal medium.

\textbf{Entertainer:} A Game Master must also be a master at improvisation. He has to be ready to handle anything that his players want to do, to resolve situations and issue rulings quickly enough to keep the pace of the game going at an entertaining clip. A Game Master is on stage, and his players are his audience.

\textbf{Inventor:} The Game Master must be the arbiter of everything that occurs in the game. All rule books, including this one, are his tools, but his word is the law. He must not antagonize the players or work to impede their ability to enjoy the game, yet neither should he favor them and coddle them. He should be impartial, fair, and consistent in his administration of the rules.

\textbf{Judge:} The Game Master's job does not end when the game session does. He must be an inventor as well. By creating NPCs, plots, magic items, spells, worlds, deities, monsters, and everything else, he propels his game's evolution forward, constantly elevating his campaign into something greater.

\textbf{Player:} Just because he's playing dozens of characters during the course of a session doesn't make him any less a player than the others who sit at the table."\ref{PFcore}

\end{quote}

\section*{Conclusion on player-game master interaction}
Based on the differences in their roles in the game and for the purpose of the game experience itself, it is most common for the game master to keep a lot of their information and planning hidden from the player till the right narrative moment. The game master needs to administrate different things in relation to a game session and so the game master's set-up is very different from the players. Therefore if any alternative application were to be added to a tabletop role-playing game session it would be beneficial for the game master to have a way to separately interact with their information unbeknown to the players but still able to as needed reveal separate parts of information to them. Potentially this will make it easier for the game master and give him better circumstances for focusing on giving the players the best possible experience.
